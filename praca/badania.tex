\chapter*{Badania}

Celem tego rozdziału jest przeprowadzenie analizy wyników klasyfikacji obrazów zwierząt dla modeli ResNet i 
ConvNeXt. Analiza obejmuje porównanie skuteczności modeli w różnych scenariuszach modyfikacji tła oraz w zależności 
od wielkości obiektu na obrazie. Przeanalizowane zostaną ogólne metryki, wyniki dla poszczególnych klas oraz wpływ 
wielkości obiektu na dokładność klasyfikacji.


\begin{table}
	\centering
	\begin{tabular}{|c|c|c|c|c|c|}
		\hline
		\textbf{Model} & \textbf{Type} & \textbf{Accuracy} & \textbf{Precision} & \textbf{Recall} & \textbf{F1-score} \\
		\hline
		ResNet & Original & 0.886500 & 0.967026 & 0.886500 & 0.922742 \\
		\hline
		ResNet & Modified & 0.697018 & 0.948539 & 0.697018 & 0.802350 \\
		\hline
		ConvNeXt & Original & 0.943300 & 0.972519 & 0.943300 & 0.956791 \\
		\hline
		ConvNeXt & Modified & 0.790873 & 0.961080 & 0.790873 & 0.866282 \\
		\hline
	\end{tabular}
	\caption{Metryki porównawcze modeli ResNet i ConvNeXt}
	\label{tab:model_comparison_metrics}
\end{table}

\begin{figure}
	\centering\includegraphics[width=.9\textwidth]{img/overall_metrics}
	\caption{Metryki dla danych oryginalnych zestawionych z danymi o zmodyfikowanych tłach}  
    \label{rys:overall_metrics}
\end{figure}

\begin{figure}
	\centering\includegraphics[width=.9\textwidth]{img/confidence_avg}
	\caption{Średnie wartości dla confidence scores}  
    \label{rys:confidence_avg}
\end{figure}


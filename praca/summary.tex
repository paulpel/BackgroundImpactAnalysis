% !TEX encoding = UTF-8 Unicode 
% !TEX root = praca.tex

\chapter*{Podsumowanie}

Analiza wyników klasyfikacji dwóch modeli – ResNet oraz ConvNeXt – wykazała, że modyfikacje tła mają znaczący wpływ na jakość klasyfikacji w obu modelach. Stwierdzono, że wielkość tego wpływu jest zależna od konkretnego modelu, klasy 
zwierzęcia, proporcji obiektu do całego zdjęcia oraz metody modyfikacji tła. Modele często popełniały błędy w przypadku zdjęć z różnorodnym tłem, co sugeruje, że w dużym stopniu polegają na informacjach zawartych w tle podczas procesu 
klasyfikacji.

Model ConvNeXt, reprezentujący nowszą architekturę, uzyskał lepsze wyniki metryk zarówno dla oryginalnych, jak i zmodyfikowanych zdjęć, jednak charakteryzował się mniejszą pewnością swoich decyzji w porównaniu do modelu ResNet. Wyizolowanie 
obiektu poprzez pozostawienie jednolitego koloru tła w niektórych przypadkach poprawiało dokładność, jednak w większości przypadków nieznacznie obniżało wartości metryk. Sugeruje to, że tło może dostarczać modeli cennych informacji, które 
pomagają w poprawnej klasyfikacji, jednak czasami wprowadza także zakłócenia, które mogą mylić model.

Efekt poprawy metryk po modyfikacji wystąpił tylko dla trzech klas i jedynie przy niektórych modyfikacjach. Wszystkie te klasy, dla których udało się uzyskać poprawę, były mylone z jedną lub dwiema innymi, bardzo podobnymi klasami. Modyfikacja 
tła poprzez przeniesienie na inną scenerię mogła pomóc modelowi w ich rozróznianiu. Wynika to prawdopodobnie z faktu, iż w zbiorze treningowym klasy prawdziwej było więcej zdjęć o podobnej scenerii do tej zmodyfikowanej niż w drugiej klasie, z 
którą była ona mylona. Dlatego w sytuacji gdy mamy klasy o bardzo podobnych cechach, modyfikacja tła może korzystnie wpłynąć na efektywność modeli.

Wyniki badania podkreślają znaczenie zróżnicowanych scenerii, obejmujących różne skale, tła i oświetlenie, w procesie uczenia modeli klasyfikacyjnych. Taka różnorodność pomaga zabezpieczyć model przed sytuacjami, w których zdjęcia są wykonane 
w nietypowej scenerii, co może prowadzić do błędnej klasyfikacji. Konkludując, zrozumienie wpływu tła na klasyfikację obrazów jest kluczowe dla poprawy wydajności modeli uczenia maszynowego i może przyczynić się do opracowania bardziej 
odpornych na zakłócenia systemów klasyfikacyjnych.


% !TEX encoding = UTF-8 Unicode 
% !TEX root = praca.tex

\chapter*{Wprowadznie}

W ostatnich latach technologie głębokiego uczenia maszynowego zrewolucjonizowały obszar przetwarzania obrazów, w tym procesy takie jak klasyfikacja i segmentacja obrazów. Fundamentalnym zagadnieniem w komputerowym rozpoznawaniu wzorców 
jest klasyfikacja obrazów, czyli proces przypisywania etykiet do obiektów przedstawionych na obrazach. Pomimo ogromnych postępów, dokładność i niezawodność modeli klasyfikacyjnych są uzależnione od wielu czynników, z których jednym z 
kluczowych jest tło obrazu.

Tło obrazu może zawierać niepotrzebne dane lub wprowadzać modele w błąd, co może powodować błędne klasyfikacje obiektów. W przypadku klasyfikacji zwierząt, obecność złożonego lub nietypowego tła może mieć znaczący wpływ na wyniki klasyfikacji. 
W związku z tym analiza wpływu tła na wyniki klasyfikacji obrazów jest bardzo ważna dla poprawy efektywności modeli.

Jednym ze sposobów rozwiązania problemu tła jest segmentacja obrazów, czyli podział obrazu na mniejsze części i przydzielenie im etykiet. Segmentacja pozwala wydzielić obiekt z tła, co może prowadzić do poprawy wyniku klasyfikacji.

Obecność zakłócających elementów w tle, zmienność oświetlenia, różnice w skalach obiektów i różnorodność danych to problemy, z którymi algorytmy klasyfikacji czy segmentacji musza sobie poradzić.  

Niniejsza praca skupia się na zbadaniu wpływu tła na klasyfikację oraz ocenę skuteczności modeli głębokiego uczenia w warunkach zmiennego tła. Przeprowadzone badania mają na celu poszerzenie wiedzy odnośnie optymalizacji modeli klasyfikacyjnych
w warunkach zmiennego i nieprzewidywalnego tła, czyli takiego jakie często występuje w rzeczywistych warunkach.

\section*{Cel pracy}

Celem niniejszej pracy jest zbadanie wpływu tła na klasyfikację zwierząt na obrazach przy użyciu zaawansowanych modeli głębokiego uczenia, takich jak ResNet i ConvNeXt. Jednym z kluczowych elementów tego badania jest ocena, w jaki sposób 
usunięcie, czy modyfikacja tła wpływają na wydajność tych modeli klasyfikacyjnych. Poprzez analizę wyników za równo przed i po modyfikacji tła, praca ta ma na celu:

\begin{itemize}
    \item \textbf{Ocena wrażliwości modeli na tło:} Sprawdzenie, jak różne rodzaje tła wpływają na dokładność (accuracy) klasyfikacji obrazów zwierząt. Analiza ta pozwoli zrozumieć, w jakim stopniu obecność tła zakłóca proces klasyfikacji i
    jakie rodzaje czy modyfikacje tła mają największy wpływ na wyniki.
    \item \textbf{Optymalizacja procesu klasyfikacji:} Stwierdzenie jakie modyfikacje tła mogą pozytywnie wpłynąć na efektywnośc klasyfikacji i okazać się przydatne przy preprocessingu danych.
    \item \textbf{Porównanie wydajności modeli:} Porównanie dwóch różnych modeli klasyfikacyjnych w celu zrozumienia jak różne architektury czy założenia modeli mogą być mniej lub bardziej odporne na zakłócenia.
\end{itemize}

\section*{Zakres pracy}

Zakres niniejszej pracy obejmuję kilka głównych aspektów, a mianowicie:

\begin{enumerate}
    \item \textbf{Przygotowanie środowiska badawczego:}
    \begin{itemize}
        \item Konfiguracja niezbędnego oprogramowania i bibliotek do przetwarzania obrazów oraz uczenia maszynowego, w celu stworzenia przystępnego środowiska do prowadzenia badań. 
        \item Ustalenie wykorzystywanych narzędzi.
    \end{itemize}
    \item \textbf{Przygotowanie danych:}
    \begin{itemize}
        \item Zebranie odpowiednich zbiorów danych zawierających obrazy zwierząt z różnorodnym tłem.
        \item Przeprowadzenie potrzebnego preprocessingu danych.
    \end{itemize}
    \item \textbf{Wykorzystanie gotowych modeli klasyfikacyjnych:}
    \begin{itemize}
        \item Wykorzystanie wcześniej już przetrenowachych modeli do klasyfikacji obrazów.
        \item Przeprowadzenie wstępnych ocen wydajności modeli na oryginalnych obrazach z różnym tłem.
    \end{itemize}
    \item \textbf{Segmentacja obrazów:}
    \begin{itemize}
        \item Wykorzystanie gotowego modelu segmentacyjnego do usunięcia tła z obrazów.
        \item Walidacja i ocena uzyskanych masek i obrazów.
    \end{itemize}
    \item \textbf{Modyfikacja tła obrazów:}
    \begin{itemize}
        \item Zastosowanie wybranych technik modyfikacji tła w celu stworzenia zestawów danych gotowych do przeprowadzania analiz klasyfikacji i możliwości porównania z wynikami na oryginalnych zdjęciach. 
    \end{itemize}
    \item \textbf{Ocena i analiza wyników:}
    \begin{itemize}
        \item Porównanie wyników klasyfikacji przed i po modyfikacjach tła za pomocą wybranych metryk.
        \item Interpretacja wyników oraz wyciągnięcie wniosków dotyczących wpływu tła na wydajność modeli 
        klasyfikacyjnych.
    \end{itemize}
    \item \textbf{Wnioski i rekomendacje:}
    \begin{itemize}
        \item Sformułowanie wniosków na podstawie przeprowadzonych eksperymentów.
        \item Propozycja potencjalnych kierunków dalszych badań oraz zastosowań praktycznych.
    \end{itemize}
\end{enumerate}
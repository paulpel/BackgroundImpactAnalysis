% !TEX encoding = UTF-8 Unicode 
% !TEX root = praca.tex

\chapter*{Wprowadznie}

W ostatnich latach technologia głębokiego uczenia maszynowego zrewolucjonizowała dziedzinę
przetwarzania obrazów, w tym klasyfikację i segmentację obrazów. Klasyfikacja obrazów, 
czyli proces przypisywania etykiet do obiektów przedstawionych na obrazach, jest fundamentalnym
zagadnieniem w komputerowym rozpoznawaniu wzorców. Pomimo znaczących postępów, istnieje wiele 
czynników, które mogą wpływać na dokładność i niezawodność modeli klasyfikacyjnych, a jednym z 
kluczowych elementów jest tło obrazu.

Tło obrazu może dostarczać zbędnych informacji lub wprowadzać modele w błąd, co może prowadzić 
do błędnej klasyfikacji obiektów. W szczególności w kontekście klasyfikacji zwierząt, obecność 
złożonego lub niestandardowego tła może znacząco wpłynąć na wyniki klasyfikacji. Dlatego analiza 
wpływu tła na wyniki klasyfikacji obrazów jest niezwykle istotna dla poprawy efektywności modeli.

Segmentacja obrazów, czyli proces podziału obrazu na mniejsze, znaczące fragmenty, jest jednym z 
podejść umożliwiających radzenie sobie z problemem tła. Dzięki segmentacji możliwe jest wydzielenie 
obiektu z tła, co może prowadzić do poprawy wyników klasyfikacji. W niniejszej pracy zostaną 
wykorzystane gotowe modele do segmentacji obrazów, w celu usunięcia tła, w celu przeprowadzenia 
późniejszych modyfikacji tła.

Wyzwania związane z klasyfikacją obrazów i segmentacją obejmują m.in. różnorodność danych, 
obecność zakłócających elementów w tle, zmienność oświetlenia oraz różnice w skalach obiektów. 
Zastosowanie zaawansowanych technik segmentacji i analizy wpływu tła może jednak znacząco poprawić
wyniki klasyfikacji.

Niniejsza praca wnosi istotny wkład do dziedziny przetwarzania obrazów, oferując nowe spostrzeżenia
na temat wpływu tła na klasyfikację oraz oceniając skuteczność różnych modeli głębokiego uczenia w 
kontekście zmiennych warunków tła. Przeprowadzone badania mają na celu pogłębienie wiedzy na temat 
optymalizacji modeli klasyfikacyjnych w złożonych i zmiennych środowiskach.

\section*{Cel pracy}

Celem niniejszej pracy jest zbadanie wpływu tła na klasyfikację zwierząt na obrazach przy 
użyciu zaawansowanych modeli głębokiego uczenia, takich jak ResNet i ConvNeXt. Kluczowym aspektem 
tego badania jest ocena, w jaki sposób usunięcie i modyfikacja tła wpływają na wydajność tych modeli 
klasyfikacyjnych. Poprzez analizę wyników przed i po modyfikacji tła, praca ta ma na celu:

\begin{itemize}
    \item \textbf{Ocena wrażliwości modeli na tło:} Sprawdzenie, jak różne rodzaje tła wpływają na 
    dokładność klasyfikacji obrazów zwierząt. Analiza ta pozwoli zrozumieć, w jakim stopniu obecność 
    tła zakłóca proces klasyfikacji i jakie rodzaje czy modyfikacje tła mają największy wpływ na wyniki.
    \item \textbf{Optymalizacja procesu klasyfikacji:} Zidentyfikowanie najlepszych praktyk i metod 
    usuwania oraz modyfikacji tła, które mogą poprawić wydajność modeli klasyfikacyjnych. Badanie to 
    pozwoli określić, które techniki segmentacji i modyfikacji tła są najbardziej efektywne w 
    kontekście różnych modeli klasyfikacyjnych.
    \item \textbf{Porównanie wydajności modeli:} Porównanie jakości klasyfikacji przy użyciu różnych 
    modeli głębokiego uczenia w kontekście zmiennych warunków tła. Analiza ta pozwoli zidentyfikować, 
    który model lepiej radzi sobie z problemem tła i jest bardziej odporny na jego zmiany.
    \item \textbf{Praktyczne implikacje:} Dostarczenie praktycznych wskazówek i rekomendacji 
    dotyczących zastosowania modeli głębokiego uczenia w zadaniach klasyfikacji obrazów w warunkach 
    rzeczywistych, gdzie tło może być zmienne i nieprzewidywalne. Wnioski z tego badania mogą być 
    użyteczne dla badaczy i praktyków zajmujących się rozpoznawaniem obrazów w różnych dziedzinach, 
    takich jak ekologia, bezpieczeństwo czy medycyna.
\end{itemize}

\section*{Zakres pracy}

Zakres pracy obejmuję kilka kluczowych aspektów, a mianowicie:

\begin{enumerate}
    \item \textbf{Przygotowanie środowiska badawczego:}
    \begin{itemize}
        \item Konfiguracja niezbędnego oprogramowania i bibliotek do przetwarzania obrazów 
        oraz uczenia maszynowego.
        \item Ustalenie parametrów eksperymentalnych i kryteriów oceny.
    \end{itemize}
    \item \textbf{Przygotowanie danych:}
    \begin{itemize}
        \item Zebranie odpowiednich zbiorów danych zawierających obrazy zwierząt z różnorodnym tłem.
        \item Przeprowadzenie potrzebnego preprocessingu danych.
    \end{itemize}
    \item \textbf{Wykorzystanie gotowych modeli klasyfikacyjnych:}
    \begin{itemize}
        \item Wykorzystanie istniejących, wytrenowanych modeli do klasyfikacji obrazów.
        \item Przeprowadzenie wstępnych ocen wydajności modeli na oryginalnych obrazach z różnym tłem.
    \end{itemize}
    \item \textbf{Segmentacja obrazów:}
    \begin{itemize}
        \item Wykorzystanie gotowego modelu segmentacyjnego do usunięcia tła z obrazów.
        \item Walidacja i ocena uzyskanych masek i obrazów.
    \end{itemize}
    \item \textbf{Modyfikacja tła obrazów:}
    \begin{itemize}
        \item Zastosowanie różnych technik modyfikacji tła w celu stworzenia zestawów danych z różnymi 
        wariantami tła.
        \item Analiza wpływu tych modyfikacji na jakość klasyfikacji.
    \end{itemize}
    \item \textbf{Ocena i analiza wyników:}
    \begin{itemize}
        \item Porównanie wyników klasyfikacji przed i po modyfikacjach tła za pomocą wybranych metryk.
        \item Interpretacja wyników oraz wyciągnięcie wniosków dotyczących wpływu tła na wydajność modeli 
        klasyfikacyjnych.
    \end{itemize}
    \item \textbf{Wnioski i rekomendacje:}
    \begin{itemize}
        \item Sformułowanie wniosków na podstawie przeprowadzonych eksperymentów.
        \item Propozycja potencjalnych kierunków dalszych badań oraz zastosowań praktycznych.
    \end{itemize}
\end{enumerate}